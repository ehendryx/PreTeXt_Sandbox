\RequirePackage{pdfmanagement-testphase}
\DeclareDocumentMetadata{uncompress,pdfversion=2.0}

\documentclass[12pt]{article}
\usepackage[tmargin=.75in,lmargin=.75in,rmargin=.75in,bmargin=.7in]{geometry}
\usepackage{graphicx}
\usepackage{caption}
\usepackage{subcaption}
\usepackage{amsfonts}
% \usepackage{float}
\usepackage{amssymb,amsthm}
\usepackage{amsmath}
\usepackage{mathtools}
\usepackage{amsopn}
\usepackage{bm}
\usepackage{setspace}
\usepackage[section]{placeins}
\usepackage{fancyhdr, lastpage}
%\usepackage{algorithmic}
\usepackage{algorithm}
\usepackage{algcompatible}
\usepackage{color}
\usepackage{multicol}
\usepackage{enumitem}
\usepackage{tikz}
\usepackage{pgfplots}

\usepackage[backend=bibtex]{biblatex}

\usepackage{tagpdf}
\tagpdfsetup{activate,paratagging,interwordspace}

\usepackage{hyperref}

\bibliography{proposal_bibliography.bib}{}
%\pagestyle{headings}
%\markright{\hfill \hfill \hfill Name: \hfill}
%\pagenumbering{gobble}

%\title{Calculus 2 \\ \vspace*{.3in}  \Large{Exam 2} \\ \vspace*{.5in} \large{October 29, 2018} \vspace*{1in}}
%\date{}

\author{\large Name: \underline{\hspace{4.5in}}}


%
\pagestyle{fancy}
\fancyhf{}
%%
%\lhead{February 17, 2017}
\chead{Applied Algebra Notes: Section 1.1}
\rfoot{\thepage{}}

%
%\cfoot{\thepage{}}

\DeclareMathOperator*{\argmin}{arg\,min}

\begin{document}
%\begin{center}
%\textbf{Introduction to Limits} % Section 1.1
%\end{center}

\subsection*{Brief Course Introduction}

Where have you seen math applied in other courses or even daily life?
\vspace{2.2in}

\noindent Using mathematical theory, notation, methods, and tools to answer practical questions is a \\

\vspace{-.12in}
\noindent form of \underline{\hspace{3.3in}}.

\vspace{.5in}
%
%\noindent \textbf{Common steps in applied math:} \hspace{1.7in} \textbf{Example:}
%\vspace{.1in}
%\begin{enumerate}
%[leftmargin=.25in]
%\item Make measurements
%\vspace{.8in}
%\item Describe how the measurements relate to one another
%\vspace{.8in}
%\item Develop a model
%\vspace{.8in}
%\item Find solution and/or write simulation\\to answer practical questions about measurements
%\vspace{.8in}
%
%\end{enumerate}
%
%\vspace{.5in}

\noindent One important skill to succeed in this class and beyond is {\bf abstraction}\index{concepts!abstraction}:
\begin{itemize}
\item the realization that the letter $x$ in an equation might represent the physical size of a cell, the cost of a sweater, the size of a population, the age of a company, etc.\\
\item the understanding that although we are interested in a specific application problem, we have to carefully apply the rules of mathematics to achieve a solution -- we solve an equation the same way whether or not it is purely a mathematical problem or represents a real-world problem\\
\item the recognition that many visibly different applied problems might be linked by {\it the same} mathematical description, and that what we learn from one system/situation could teach us about another -- even though we might not recognize any direct real-world similarities between the two systems
\end{itemize}


\vspace{.3in}

\subsection*{Section 1.1: Linear Models}

\noindent A model describes a relationship between variables. This relationship can be represented in a variety of ways:


\begin{itemize}
\item with a table of values\\
\item with a graph\\
\item with an equation\\
\end{itemize}

\vspace{.1in}
\noindent \textbf{Ex:} A taxi out of Dulles Airport in Washington, DC, charges a traveler an initial fee of \$2.00, plus \$1.50 for each mile traveled. 

\begin{enumerate}
\item Complete the \textbf{table of values} showing the charge, $C$, for a trip of $n$ miles.

\begin{table}[h!]
\begin{center}
\renewcommand{\arraystretch}{1.7}
\begin{tabular}{|c|c|c|c|c|c|c|}
\hline
$n$ &	\hspace*{.2in} 0 \hspace*{.2in} &	\hspace*{.2in} 5 \hspace*{.2in} &	\hspace*{.2in} 10 \hspace*{.2in} & \hspace*{.2in} 15 \hspace*{.2in} & \hspace*{.2in} 20 \hspace*{.2in} &  \hspace*{.2in} 25 \hspace*{.2in}\\
\hline
$C$ & & & & & & \\
\hline	
\end{tabular}
\end{center}
\end{table}			

\item \textbf{Graph} the relationship.
%
%\hspace*{3in}
%\begin{tikzpicture}[scale=.59]
%  \begin{axis}[width=0.8\textwidth,
%       height=0.8\textwidth,grid=major, ymin=-5,ymax=60,xmax=40,xmin=-5, ticks=none]
%\end{axis}
%\end{tikzpicture}

%
%\begin{tikzpicture}
%   \draw[step=5cm,color=gray] (0,0) grid (40,60);
%\end{tikzpicture}

%\vspace{3in}
\begin{center}
%\hspace*{-.6in}
\begin{tikzpicture}[scale=.59]
  \begin{axis}[width=0.7\textwidth,
       height=0.8\textwidth,axis lines=center,grid=major, ymin=-2,ymax=62,xmax=42,xmin=-2,axis line style={latex-latex}, ticks=none]
\end{axis}
\end{tikzpicture}
\end{center}

\item Write an \textbf{equation} for the charge, $C$, in terms of the number of miles traveled, $n$.


\vspace{1in}

\item What is the charge for a trip to Mount Vernon, 35 miles from the airport? Illustrate the answer on your graph.


\vspace{1in}




\item If a ride to the National Institutes of Health (NIH) costs \$42.50, how far is it from the airport to the NIH? Illustrate the answer on your graph.


\vspace{1.2in}
\end{enumerate}

\noindent The \underline{\hspace{1.2in}} of a graph are the points where the graph crosses the axes.\\

\begin{itemize}
\item The  \underline{\hspace{1.2in}} is the point $(0,y)$ found by setting   \underline{\hspace{1in}} and solving for y.\\
\item The  \underline{\hspace{1.2in}} is the point $(x,0)$ found by setting  \underline{\hspace{1in}} and solving for x.
\end{itemize}

\vspace{.3in}
\noindent \textbf{Ex:} Find the intercepts of the graph in the airport example above. What do these intercepts represent?

\vspace{2.2in}
\noindent \hrulefill

\noindent \textbf{Graphing a Line Using the ``Intercept Method'':}
\begin{enumerate}
\item Find the $x$- and $y$-intercepts.
\item Plot the intercepts.
\item Choose a value of $x$ and find a third point on the line to make it a bit easier to make the line straight.
\item Draw a line through the points.
\end{enumerate}
\noindent \hrulefill

\vspace{.3in}

\noindent In the airport example, the model was of the form 
\vspace{1.2in}

\noindent Models of this form are called \underline{\hspace{3in}}. 
\vspace{.2in}

\noindent Linear equations can also be written in what is sometimes called \textbf{general form:}
\vspace{1.2in}

\newpage 
\noindent \textbf{Ex:} Five pounds of body fat is equivalent to 16,000 calories. Carol can burn 600 calories per hour bicycling and 400 calories per hour swimming.  

\begin{enumerate}
\item Write an equation in general form that relates the number of hours, $x$, of cycling and the number of hours, $y$, of swimming Carol needs to perform to burn 16,000 calories. 
\vspace{1in}

\item Find the intercepts and use them to sketch the graph of the linear equation. 
\vspace{.4in}


\hspace*{3in}
\begin{tikzpicture}[scale=.59]
  \begin{axis}[width=0.9\textwidth,
       height=0.9\textwidth,axis lines=center,grid=major, ymin=-1.5,ymax=12.5,xmax=12.5,xmin=-1.5,axis line style={latex-latex}, ticks=none]
\end{axis}
\end{tikzpicture}

\vspace{.4in}
\item What do the intercepts tell us about Carol’s exercise program?  
\vspace{1.5in}

\item Is the graph you drew in part (2) \textit{increasing} or \textit{decreasing}?
\vspace{1.5in}
\end{enumerate}
 
 
\noindent \textbf{Ex:}
Frank plants a dozen corn seedlings, each 6 inches tall.  With plenty of water and sunlight they will grow approximately 2 inches per day.  

\begin{enumerate}
\item Complete the table of values for the height, $h$, of the seedlings after $t$ days.  

 
 \begin{table}[h!]
\begin{center}
\renewcommand{\arraystretch}{1.7}
\begin{tabular}{|c|c|c|c|c|c|c|}
\hline
$t$ &	\hspace*{.2in} 0 \hspace*{.2in} &	\hspace*{.2in} 5 \hspace*{.2in} &	\hspace*{.2in} 10 \hspace*{.2in} & \hspace*{.2in} 15 \hspace*{.2in} & \hspace*{.2in} 20 \hspace*{.2in} &  \hspace*{.2in} 25 \hspace*{.2in}\\
\hline
$h$ & & & & & & \\
\hline	
\end{tabular}
\end{center}
\end{table}	

\item Write an equation for the height of the seedlings in terms of the number of days since they were planted.
\vspace{1.1in}

\item Find the intercepts of the graph.
\vspace{1.4in}

\item Graph the equation.
\vspace{2.5in}
%\hspace*{3in}
%\begin{tikzpicture}[scale=.59]
%  \begin{axis}[width=0.8\textwidth,
%       height=0.8\textwidth,grid=major, ymin=-1.5,ymax=12.5,xmax=12.5,xmin=-.5, ticks=none]
%\end{axis}
%\end{tikzpicture}

\vspace{.1in}


\item What do the intercepts tell us about the problem?
\vspace{2in}

\end{enumerate}



\end{document}